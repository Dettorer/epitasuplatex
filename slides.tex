\documentclass{beamer} % beamer is the document class for presentation slides

\usepackage{hyperref} % commands such as \href, \url, etc.
\usepackage[english,french]{babel} % load typography rules for the french language
\usepackage{biblatex} % bibliography handling
\usepackage{csquotes} % biblatex extensions for non-english languages
\usepackage{tipa} % International Phonetic Alphabet symbols and helpers

\usetheme{Warsaw}
\useoutertheme{infolines} % Display the slides number in the footline (and a smaller header)

% Add a tableofcontent slide before each subsection, highlighting the section we're at
\AtBeginSection[]{
    \frame{\tableofcontents[currentsection]}
}

% define the \ipa command that switches to english typography rules before
% calling \textipa (which gets confused by french typography rules)
\newcommand{\ipa}[1] {
    \foreignlanguage{english}{\textipa{#1}}
}

% Bibliography configuration
\bibliography{references}

\title{Écriture de documents avec \LaTeX{}}
\subtitle{\texorpdfstring{\small{\emph{D'après une conférence de Didier Verna}\cite{latexDV}}}
                         {D'après une conférence de Didier Verna}}
\institute{EPITA}
\author{
    % The authors information appear in the first page of the document but also
    % in the pdf's metadata. A lot of stuff isn't allowed in the metadata
    % (links, line breaks, etc.) so we must define two alternatives, one for the
    % document page and one for the metadata, this is done with the
    % \texordpdfstring command
    \texorpdfstring
    { % Nicely formatted author information
        Paul Hervot\\
        \href{mailto:paul.hervot@epita.fr}{\nolinkurl{paul.hervot@epita.fr}}
    }{ % A simpler version for the pdf's metadata
        Paul Hervot
    }
}
\date{3 septembre 2019}

\begin{document}
\frame{\titlepage{}}

\section{Introduction}
\subsection{\TeX}
\begin{frame}{\TeX}
    \begin{description}
        \item[Objectif :] mise en forme typographique entièrement informatique
        \item[Création :] Donald Knuth, 1977-1989
        \item[Prononciation :] \ipa{[tEx]} (symbolise les lettres $\tau\varepsilon\chi$)
    \end{description}

    \begin{figure}[position]
        \includegraphics[height=0.5\textheight]{img/knuth}
        \caption{Donald Knuth, en 2005}
    \end{figure}
\end{frame}

\subsection{\LaTeX}
\begin{frame}{\LaTeX}
    \begin{description}
        \item[Prononciation :] \ipa{["lA:tEx]} ou \ipa{["leI:tEx]}
    \end{description}
\end{frame}

\section{Conclusion}
\subsection{Bibliography}
\begin{frame}[allowframebreaks]{Bibliographie}
    \printbibliography{}
\end{frame}
\end{document}
