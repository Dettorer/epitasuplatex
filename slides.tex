\documentclass[usenames,dvipsnames]{beamer} % beamer is the document class for presentation slides

% languages typesetting rules
\usepackage{polyglossia} % handle multiple languages typesetting rules
\setdefaultlanguage{french} % this presentation is in french
\setotherlanguages{english,greek} % this presentation sometimes uses english and greek texts
\newfontfamily\greekfont[Script=Greek]{Linux Libertine O} % we need a special font for greek text
\newfontfamily\greekfontsf[Script=Greek]{Linux Libertine O} % same, but sans serif

% links configuration
\usepackage{hyperref} % commands such as \href, \url, etc.
\hypersetup{
    colorlinks=true,
    linkcolor=,
    urlcolor=Brown
}

% bibliography configuration
\usepackage[backref=true]{biblatex} % bibliography handling
\newcommand{\titlecite}[1] {
    % Insert the (short) title of the citation, and a reference to it
    \citetitle{#1}\cite{#1}
}

\usepackage{csquotes} % biblatex extensions for non-english languages
\usepackage{tipa} % International Phonetic Alphabet symbols and helpers
\usepackage{caption} % more control over captions
\usepackage{ccicons} % icons for Creative-Commons licenses

\usetheme{Warsaw}
\useoutertheme{infolines} % Display the slides number in the footline (and a smaller header)

% Add a tableofcontent slide before each subsection, highlighting the section we're at
\AtBeginSection[]{
    \frame{\tableofcontents[currentsection]}
}

% define the \ipa command that switches to english typography rules before
% calling \textipa (which gets confused by french typography rules)
\newcommand{\ipa}[1] {
    \selectlanguage{english}
    \textipa{#1}
}

% Bibliography configuration
\bibliography{references}

\title{Écriture de documents avec \LaTeX{}}
\subtitle{\texorpdfstring{\small{\emph{D'après une conférence de Didier Verna}\cite{latexDV}}}
                         {D'après une conférence de Didier Verna}}
\titlegraphic{\large{\href{https://creativecommons.org/licenses/by-sa/4.0/}{\ccbysa}}}
\institute[]{EPITA Strasbourg}
\author[Paul Hervot]{
    % The authors information appear in the first page of the document but also
    % in the pdf's metadata. A lot of stuff isn't allowed in the metadata
    % (links, line breaks, etc.) so we must define two alternatives, one for the
    % document page and one for the metadata, this is done with the
    % \texordpdfstring command
    \texorpdfstring
    { % Nicely formatted author information
        Paul Hervot\\
        \href{mailto:paul.hervot@epita.fr}{\nolinkurl{paul.hervot@epita.fr}}
    }{ % A simpler version for the pdf's metadata
        Paul Hervot
    }
}
\date{3 septembre 2019}

\begin{document}
\frame{\titlepage{}}

\section{Introduction}
\subsection{\TeX}
\begin{frame}{\TeX}
    \begin{description}
        \item[Nom :] symbolise les lettres \textgreek{τεχ}, abbréviation de \textgreek{τέχνη}
        \item[Objectif :] mise en forme typographique entièrement informatique
        \item[Création :] Donald Knuth, 1977-1989
        \item[Prononciation :] \ipa{[tEx]}
    \end{description}

    \begin{figure}[position]
        \includegraphics[height=0.5\textheight]{img/knuth}
        \caption*{Donald Knuth, en 2005\cite{knuthpic}}
    \end{figure}
\end{frame}

\subsection{\LaTeX}
\begin{frame}{\LaTeX}
    \begin{description}
        \item[Nom :] abbréviation de Lamport-\TeX
        \item[Objectif :] ensemble de macros (raccourcis) pour \TeX
        \item[Création :] Leslie Lamport, 1983
        \item[Prononciation :] \ipa{["lA:tEx]} ou \ipa{["leI:tEx]}
    \end{description}

    \begin{figure}[position]
        \includegraphics[height=0.5\textheight]{img/lamport}
        \caption*{Leslie Lamport, vers 2014}
    \end{figure}
\end{frame}

\begin{frame}{Intérêts aujourd'hui}
    \begin{itemize}
        \item édition mathématique ;
        \item gestion de bibliographie ;
        \item écriture sous forme de code source distribuable ;
        \item totalement libre, extensible et personnalisable ;
        \item très utilisé dans le milieu de la recherche.
    \end{itemize}
\end{frame}

\begin{frame}{Obtenir \LaTeX}
    \begin{description}
        \item[Multi-plateforme :] \href{http://tug.org/texlive/}{\TeX{}Live}.
            Standard \emph{de facto}, disponible dans toutes les destribution
            Linux.
        \item[Windows :] \href{https://miktex.org/}{Mik\TeX}.
        \item[MacOS :] \href{https://www.tug.org/mactex/}{Mac\TeX}, basé sur \TeX{}Live.
        \item[En ligne :] \href{https://www.overleaf.com/}{Overleaf}, payant avec version d'essai.
    \end{description}

    \vspace{1cm}
    Il existe aussi de nombreux IDEs\footnote{Environnement de Développement
    Intégré} comme \href{https://www.xm1math.net/texmaker/}{\TeX{}Maker} et des
    plugins pour éditeurs comme \href{https://github.com/lervag/vimtex}{vimtex}.
    \LaTeX{} étant libre, il existe un très grand nombre d'outils autour.
\end{frame}

\section{Conclusion}
\subsection{Autres ressources et références}
\begin{frame}{Autre documentation}
    \begin{description}
        \item[Livres]{
            \begin{itemize}
                \item \titlecite{lamport1994latex}
                \item \titlecite{goossens1997latex}
                \item \titlecite{rolland1999latex}
            \end{itemize}
        }
        \item[En ligne]{
            \begin{itemize}
                \item \titlecite{oetiker2011not}
                \item \url{http://www.latex-project.org}
                \item \url{http://www.ctan.org} (le "\textenglish{Comprehensive
                    \TeX{} Archive Network}", documentation complète de toutes
                    les extensions)
                \item \url{https://tex.stackexchange.com} (forum de
                    question/réponses)
                \item \url{https://en.wikibooks.org/wiki/LaTeX}
            \end{itemize}
        }
    \end{description}
\end{frame}
\begin{frame}[allowframebreaks]{Bibliographie}
    \printbibliography{}
\end{frame}
\end{document}
